\documentclass[a4paper,12pt]{article}
\usepackage{HomeWorkTemplate}

\usepackage[utf8]{inputenc}
\usepackage[]{babel}

\setlength{\parindent}{4em}
\setlength{\parskip}{0.5em}

\renewcommand{\baselinestretch}{1.5}


\usepackage[utf8]{inputenc}
\usepackage{lmodern, textcomp}
\usepackage{circuitikz}
\usepackage[shortlabels]{enumitem}
\usepackage{hyperref}
\usepackage{tikz}
\usepackage{amsmath}
\usepackage{amssymb}
\usepackage{tcolorbox}
\usepackage{graphicx}
\usepackage{xepersian}
\settextfont{XB Niloofar}
\usetikzlibrary{arrows,automata}
\usetikzlibrary{circuits.logic.US}
\usepackage{changepage}
\newcounter{problemcounter}
\newcounter{subproblemcounter}
\setcounter{problemcounter}{1}
\setcounter{subproblemcounter}{1}
\newcommand{\problem}[1]
{
	\subsection*{
		پرسش
		\arabic{problemcounter} 
		\stepcounter{problemcounter}
		\setcounter{subproblemcounter}{1}
		#1
	}
}
\newcommand{\subproblem}{
	\textbf{\harfi{subproblemcounter})}\stepcounter{subproblemcounter}
}


\begin{document}
\handout
{اصول پردازش تصویر}
{دکتر مصطفی کمالی تبریزی}
{نیم‌سال اول 1399\lr{-}1400}
{اطلاعیه}
{سیدعلیرضا خادم}
{97100398}
 {تمرین سری اول - سوال اول}
\section*{موارد لازم.}
برای اینکه کد 
$ q_1.py $
به درستی اجرا شود، لازم است که تصویرِ ورودی با نامِ
\lr{Dark.jpg}
در مسیرِ 
\lr{EX1\_Q1/images/}
قرار بگیرد.
\section*{روند کلی حل.}
با توجه به اینکه تصوری که به عنوان ورودیِ سوال داده شده است یک تصویرِ تیره است که بسیاری از 
\lr{pixel}
هایِ آن دارای 
\lr{intensity}
کمی هستند روشن تر کردن تصویر، می‌تواند با استفاده از روش‌های 
\lr{Image Enhancement}
مبتنی بر 
\lr{Point Operations}
انجام شود.\\
در این سوال با ابتدا 
\lr{histogram}
عکس ورودی را رسم کردیم، بعد با توجه به اینکه مشاهده شد که
\lr{histogram}
عکس توزیعِ مناسبی از 
 \lr{intensity}
 را نشان می‌داد و تنها مقدارِ
 \lr{intensity}
 ها کم بود تصمیم بر این شد که تنها مقدار این 
 \lr{intensity}
 ها در ثابتِ 4 ضرب شود. دلیل عدم استفاده از 
 \lr{Log-Transformation}
 و 
 \lr{Gamma-Transformation}
 این بود که این 
 \lr{transformation}
 ها هستوگرام عکس رو به صورتی تغییر میدادند که باعث کم شدنِ
 \lr{contrast} 
عکس و مصنوعی شدن آن می‌شدند.\\ برایِ مشاهده‌ی این موضوع می‌توانید نمودارهایی که در عکسِ
\lr{res.png}
در مسیر
\lr{EX1\_Q1/results/}
است را مشاهده کنید که به ترتیب از بالا به پایین نتیجه‌ی ضرب عکس در ثابتِ 4 ، 
$ Gamma-Transformation :gamma = 0.25 $
و
$Log-Transformation : coefficient = 1.2 $
است و از چپ به راست 
\lr{channel}
هایِ آبی، سبز و قرمز را نشان میدهد. (نمودار نارنجی رنگ هیستوگرام نرمال شده و نمودار آبی رنگ تابع توزیع تجمعی نرمال شده را نشان میدهند)
\section*{توضیح کد.}
\subsection*{$\circ$ utilities.py}
\subsection*{\text{gamma\_transformation}}
این تابع یک تصویر و یک 
\lr{gamma}
به عنوان ورودی می‌گیرد و نیتیجه‌یِ اعمال 
\lr{Gamma-Transformation}
با این 
\lr{gamma}
را روی عکس برمی گرداند.
\subsubsection*{log\_transformation}
این تابع یک تصویر و یک 
\lr{coefficient}
به عنوان ورودی می‌گیرد و نیتیجه‌یِ اعمال 
\lr{Log-Transformation}
با ضریب
\lr{coefficient}
 روی عکس را برمی گرداند.
 \subsubsection*{get\_histogram}
 یک تصویر به عنوان ورودی می‌گیرد و هیستوگرامِ کانال‌هایِ 
 \lr{B}
 ،
\lr{G}
و
\lr{R}
را به صورت یک سه‌تایی برمی‌گرداند.
\subsubsection*{get\_cdf}
یک تصویر به عنوان ورودی می‌گیرد و تابع توزیع تجمعیِ کانال‌هایِ 
\lr{B}
،
\lr{G}
و
\lr{R}
را به صورت یک سه‌تایی برمی‌گرداند.
\subsubsection*{plot\_cdf , plot\_pdf}
این توابع برای 
\lr{plot}
کردن هستوگرام‌ها و تابع توزیع تجمعی‌ها و همچنین جلوگیری از تکرار کد پیاده‌سازی شده است.
\subsection*{$\circ$ q1.py}
در این فایل تصویر لود شده و بعد سه 
\lr{tranformation}
روی تصویر اعمال شده است که تنها اولی مد نظر ما بوده و دو
\lr{transformation}
  دیگر برای استدال عدم استفاده از 
\lr{Log-Transformation}
و 
\lr{Gamma-Transformation}
انجام شده. بعد 
\lr{cdf}
و 
\lr{pdf}
مربوط به این سه نتیجه آماده‌یِ نمایش می‌شود و در نهایت 
\lr{result\_1}
که نتیجه‌یِ مطلوب ما است به عنوان پاسخ تمرین ذخیره می‌شود و بعد نمودار‌ها برایِ مقایسه سه 
\lr{transformation}
ای که روی تصور اعمال شده بود، نمایش داده می‌شود.

\end{document}
