\documentclass[a4paper,12pt]{article}
\usepackage{HomeWorkTemplate}

\usepackage[utf8]{inputenc}
\usepackage[]{babel}

\setlength{\parindent}{4em}
\setlength{\parskip}{0.5em}

\renewcommand{\baselinestretch}{1.5}


\usepackage{caption}
\usepackage{subcaption}
\usepackage{graphicx}
\usepackage{float}
\usepackage[utf8]{inputenc}
\usepackage{lmodern, textcomp}
\usepackage{circuitikz}
\usepackage[shortlabels]{enumitem}
\usepackage{hyperref}
\usepackage{tikz}
\usepackage{amsmath}
\usepackage{amssymb}
\usepackage{tcolorbox}
\usepackage{graphicx}
\usepackage{xepersian}
\settextfont{XB Niloofar}
\usetikzlibrary{arrows,automata}
\usetikzlibrary{circuits.logic.US}
\usepackage{changepage}
\newcounter{problemcounter}
\newcounter{subproblemcounter}
\setcounter{problemcounter}{1}
\setcounter{subproblemcounter}{1}
\newcommand{\problem}[1]
{
	\subsection*{
		پرسش
		\arabic{problemcounter} 
		\stepcounter{problemcounter}
		\setcounter{subproblemcounter}{1}
		#1
	}
}
\newcommand{\subproblem}{
	\textbf{\harfi{subproblemcounter})}\stepcounter{subproblemcounter}
}


\begin{document}
\handout
{اصول پردازش تصویر}
{دکتر مصطفی کمالی تبریزی}
{نیم‌سال اول 1399\lr{-}1400}
{اطلاعیه}
{سیدعلیرضا خادم}
{97100398}
 {تمرین سری اول - سوال چهارم}
\section*{موارد لازم.}\label{n}
برای اینکه کد 
$ q_4.py $
به درستی اجرا شود، لازم است که تصاویر
\lr{Dark.jpg}
و
\lr{Pink.jpg}
 در مسیرِ 
\lr{EX1\_Q4/images/}
قرار بگیرد.
\section*{روند کلی حل.}
حل این سوال به صورت تکنیکال انجام می‌شود و نحوه انجام آن از 
\textbf{توضیح کد}
قابل فهم است.
\section*{توضیح کد.}
\subsection*{$\circ$ utilities.py}
\subsubsection*{find\_nearst}
این تابع یه آرایه و یه 
\lr{value}
 را به عنوان ورودی می‌گیرد و اندیسِ نزدیکترین مقدار در آرایه به این
 \lr{value}
 را برمی‌گرداند.
 \subsubsection*{histogram}
 این تابع یک آرایه (عکس) را به عنوان ورودی می‌گیرد و هیستوگرام آن را به عنوان خروجی برمی‌گرداند.
 \subsubsection*{one\_channel\_histogram\_specified}
 این تابع دو چنل را به عنوانِ
 \lr{target\_channel}
 و 
 \lr{input\_channel}
 می‌گیردو ابتدا هیستگرام این دو را با استفاده از تابع
 \lr{histogram}
 محاسبه می‌کند و بعد تابع توزیع تجمعی این دو هستوگرام را با استفاده از تابع 
 \lr{cumsum}
 محاسبه می‌کند و به ترتیب  در
  \lr{cdf\_target}
 و 
 \lr{cdf\_input}
 می‌ریزد بعد هر دو را نرمال می‌کند. در ادامه در یک لوپ که از 255 تا 0 است به ازای هر 
 \lr{intensity}
 مقدارش را در تابع توزیع تجمعی نرمال شده‌یِ 
 \lr{input\_channel}
 نگاه می‌کند و نزدیکترین مقدار را در تابع توزیع تجمعیِ
 \lr{target\_channel}
 با استفاده از تابع 
 \lr{find\_nearest}
 سرچ می‌زنیم و اندیسی که تابع 
 \lr{find\_nearest}
 برمی‌گرداند را به عنوان 
 \lr{intensity}
 ای که قرار است به آن مپ شود در نظر می‌گیریم. و در نهایت 
  \lr{output}
  را که 
  \lr{histogram specified}
  شده‌یِ 
  \lr{input\_channel}
  است را به عنوان خروجی برمی‌گردانیم.
  \subsubsection*{histogram\_specified}
  این تابع یک عکس را به عنوان 
  \lr{input\_image}
  و یک عکس را به عنوان
  \lr{target\_image}
  ورودی می‌گیرد و کانال‌های هر یک را با استفاده از تابع
  \lr{split}
  و هیستگرامِ کانال‌هایِ 
  \lr{input\_image}
  را با استفاده از تابع \\
  \lr{one\_channel\_histogram\_specified}
  به هیستوگرامِ کانال متانظرشان تبدیل می‌کنیم. در نهایت هم سه کانال حال از
  \lr{one\_channel\_histogram\_specified}
  را با هم مرج می‌کنیم و به عنوان خروجی می‌دهیم.
\subsection*{$\circ$ q4.py}
در این فایل ابتدا تصویر 
\lr{original}
و 
\lr{target}
لود شده است و بعد با استفاده از تابع 
\lr{histogram\_specified}
هیستگرام عکس اصلی 
\lr{specified}
شده و نتیجه در مسیر
\lr{EX1\_Q4/results/}
  ذخیره شده است.
\end{document}
