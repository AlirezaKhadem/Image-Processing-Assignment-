\documentclass[a4paper,12pt]{article}
\usepackage{HomeWorkTemplate}

\usepackage[utf8]{inputenc}
\usepackage[]{babel}

\setlength{\parindent}{4em}
\setlength{\parskip}{0.5em}

\renewcommand{\baselinestretch}{1.5}


\usepackage{caption}
\usepackage{subcaption}
\usepackage{graphicx}
\usepackage{float}
\usepackage[utf8]{inputenc}
\usepackage{lmodern, textcomp}
\usepackage{circuitikz}
\usepackage[shortlabels]{enumitem}
\usepackage{hyperref}
\usepackage{tikz}
\usepackage{amsmath}
\usepackage{amssymb}
\usepackage{tcolorbox}
\usepackage{graphicx}
\usepackage{xepersian}
\settextfont{XB Niloofar}
\usetikzlibrary{arrows,automata}
\usetikzlibrary{circuits.logic.US}
\usepackage{changepage}
\newcounter{problemcounter}
\newcounter{subproblemcounter}
\setcounter{problemcounter}{1}
\setcounter{subproblemcounter}{1}
\newcommand{\problem}[1]
{
	\subsection*{
		پرسش
		\arabic{problemcounter} 
		\stepcounter{problemcounter}
		\setcounter{subproblemcounter}{1}
		#1
	}
}
\newcommand{\subproblem}{
	\textbf{\harfi{subproblemcounter})}\stepcounter{subproblemcounter}
}


\begin{document}
\handout
{اصول پردازش تصویر}
{دکتر مصطفی کمالی تبریزی}
{نیم‌سال اول 1399\lr{-}1400}
{اطلاعیه}
{سیدعلیرضا خادم}
{97100398}
 {تمرین سری اول - سوال دوم}
\section*{موارد لازم.}\label{n}
برای اینکه کد 
$ q_2.py $
به درستی اجرا شود، لازم است که تصویرِ گل‌های زرد و تصویر گل‌های صورتی به ترتیب با نامِ‌های
\lr{02.jpg}
و
\lr{03.jpg}
در مسیرِ 
\lr{EX1\_Q2/images/}
قرار بگیرد. بعد از اجرایِ کد از شما خواسته می‌شود که نامِ تصویری که قصد دارید به عنوان ورودی به این کد داره شود تا تغییرات ذکر شده در سوال صورت 2 روی آن اعمال شود را همراه با فرمت آن وارد کنید، در این قسمت شما با توجه به نیاز بایستی 
\lr{02.jpg}
را برایِ تصویر گل‌های زرد و یا
\lr{03.jpg}
را برای تصویر گل‌های صورتی وارد کنید.\\ بعد از وارد کردن نام تصویرِ ورودی،  عبارتِ 
\lr{"Do you want to use new sampling ?! Y/N"}
برای شما در کنسول نشان داده خواهد شد که با وارد کردن حرف 
\lr{y}
یا 
\lr{Y}
این اجازه به شما داده می‌شود که در تصویری که برای شما نمایش داده شده و در نیمه‌یِ چپ تصویر با کلیک چپِ موس خود پیکسل‌هایی که میخواهید رنگ آن‌ها تغییر کند را انتخاب کنید. برای مثال در تصویر گل‌های زرد می‌تواند با کلیک چپ روی گل‌هایِ رنگ زرد، یک بازه‌ای از رنگ زرد را به عنوان رنگی که میخواهد تغییر کند انتخاب کنید.\\
در صورت وارد کردن 
 \lr{n}
 یا 
 \lr{N}
 از نمونه‌گیری‌ای که قبلا انجام شده و در فایلِ مربط به هر عکس و در مسیرِ
 \lr{EX\_1Q2/samples/}
 استفاده خواهد شد.
 بعد از مشخص شدنِ اینکه نمونه گیری به چه صورتی میخواهد انجام شود همان طور که در تصاویر زیر مشاهده می‌کنید، می‌توانید
 \lr{Trackbar}
 ای که در بالایِ پنجره قرار دارد را روی 
 \lr{Hue}
 تارگت قرار دهیم ( عدد متناظر با 
 \lr{Hue}
که میخواهیم رنگ گل‌ها به آن
\lr{Hue}
تبدیل شود
 ) 
 و آن قسمت‌هایی از عکس که نمیخواهیم تغییر رنگ در آن قسمت‌ها اتفاق بیافتد را با کشیدن مستطیلی سبز رنگ مشخص می‌کنیم( کشیدن مستطیل را با قرار دادن موس در گوشه‌یِ بالا سمتِ چپ مستطیل و نگه داشتن کلیک راست موس و کشیدن موس تا گوشه‌یِ پایین سمت راست انجام می‌دهیم).
 \begin{figure}
	\centering
 	\begin{subfigure}{0.9\textwidth}
 		\centering
 		\includegraphics[width=\textwidth]{02_data.jpg}
 	\end{subfigure}
 	\vspace{1cm}
 	\begin{subfigure}{0.9\textwidth}
 		\centering
		\includegraphics[width=\textwidth]{03_data.jpg}
	\end{subfigure}
 \end{figure}
در هر لحظه می‌توان با فشردن کلید
\lr{u}
روی کیبورد، تغییراتی که روی تصویر اعمال شده است را در نیمه‌یِ سمت راستِ پنجره مشاهده کرد.
\section*{روند کلی حل.}
ایده و روند اصلی حل این مسئله به این صورت است که ما با روشی (که این روش در اینجا نمونه گیری است) بازهایی برایِ
\lr{Hue}
،
  \lr{Saturation}
  و 
  \lr{Value}
  شئ‌هایی در تصویر که میخواهیم رنگ آن‌ها را تغییر دهیم، بیابیم و بعد آن قسمت‌هایی از تصویر که مقدار آن‌ها در این بازه ها می‌افتد را پیدا کنیم و 
  \lr{Hue}
  آنها را به گونه‌ای تغییر دهیم که رنگ آن‌ها به رنگ هدف تبدیل بشه. در این میان شاید شئ‌های دیگر هم در تصویر باشند که مقدارِ
  \lr{Hue}
  ،
  \lr{Saturation}
  و 
  \lr{Value}
  آنها در بازه‌ای که ما تعیین کرده‌ایم قرار بگیرد اما نخواهیم رنگ آن‌ها تغییر کند که این مشکل را ما تا حدودی با مشخص کردن نقاطی که نمی‌خواهیم تغییر رنگ داشته باشیم، با رسم مستطیلی که آن نقاط را بپوشاند، رفع کرده‌ایم.
  

\section*{توضیح کد.}
\subsection*{$\circ$ utilities.py}

\subsubsection*{load\_image}
این تابع نام تصویر را می‌گیرد و آن را از مسیر
\lr{EX\_1Q2/images/}
لود می‌کند.
\subsubsection*{get\_image\_name}
این تابع با چاپ کردنِ عبارتِ
 \lr{"please enter name of image with its format"}
 از کاربر میخواهد تا نام عکسی که به عنوان ورودی قرار است به برنامه داده شود را همراه با فرمت تایپ کند.
 \subsubsection*{load\_sample}
 این تابع با چاپ عبارتِ
 \lr{"Do you want to use new sampling ?! Y/N"}
 از شما میخواهد که نحوه نمونه گیری را با توجه به توزیحاتی که در قسمت 
 \textbf{موارد لازم}
 گفته شد مشخص کنید. در صورت وارد کردن 
 \lr{n}
 یا 
 \lr{N}
 این تابع نمونه‌گیری‌ای که قبلا انجام شده و در فایلِ مربط به هر عکس و در مسیرِ
 \lr{EX\_1Q2/samples/}
 قرار دارد را، لود می‌کند.و اگر 
 \lr{y}
 یا
 \lr{Y}
 وارد شود 
 \lr{sampling = True}
 شده و آرایه‌ای خالی به عنوان 
 \lr{sample}
 برمی‌گرداند.
  \subsubsection*{save\_sample}
  این تابع نام تصویر را به عنوان ورودی می‌گیرد و 
  \lr{sample}
  را به فرمت خاصی که وابسته به نام تصویر ورودی است در مسیرِ
  \lr{EX\_1Q2/samples/}
  با فرمت
  \lr{.txt}
  ذخیره می‌کند.
  \subsubsection*{save\_image}
  این تابع تصویر را در مسیرِ
  \lr{EX\_1Q2/images/}
  با اسمی که در صورت سوال خواسته شده بود ذخیره می‌کند.
  \subsubsection*{convert2hsv}
  این تایع تصویری را در فضایِ
  \lr{BGR}
   به عنوان ورودی می‌گیرد و تبدیل شده‌یِ آن را به فضایِ
  \lr{HSV}
  باز می‌گرداند.
  \subsubsection*{mouse\_handler}
  این تابع ایونت هایِ مربوط به موس را هندل می‌کند. به این صورت که اگر کاربر قسمت‌هایی که میخواهد رنگ آن ها را تغییر نکند را با موس تعیین کند،‌این تابع با فراخوانی تابع 
  \lr{select}
  این امکان را فراهم می‌آورد. همچنین اگر 
  \lr{sampling = True}
  باشد و کاربر اجازه نمونه‌گیری را داشته باشد، این تابع با فراخوانی تابع
  \lr{get\_sample}
  این امکان رو فراهم می‌آورد.
  \subsubsection*{select}
  همان طور که در توضیح تابع قبلی گفته شد این تابع در پیاده‌سازیِ تابع
  \lr{mouse\_handler}
  استفاده می‌شود و این امکان رو به کاربر می‌دهد که آن قسمت‌هایی از عکس که نمیخواهد تغییر رنگ در آن قسمت‌ها اتفاق بیافتد را با کشیدن مستطیلی سبز رنگ مشخص کند( کشیدن مستطیل را با قرار دادن موس در گوشه‌یِ بالا سمتِ چپ مستطیل و نگه داشتن کلیک راست موس و کشیدن موس تا گوشه‌یِ پایین سمت راست انجام می‌دهیم).
  \subsubsection*{get\_sample}
  همان طور که در توضیح تابع 
  \lr{mouse\_handler}
   گفته شد این تابع در پیاده‌سازیِ تابع
  \lr{mouse\_handler}
  استفاده می‌شود و این امکان رو به کاربر می‌دهد که با کلیک چپ موس خود به نمونه، سه‌تاییِ
  $ (Hue
  ,
  Saturation
  ,
  Value) $
  آن نقطه‌ای که روی آن کلیک شده را اضافه کند.
  \subsubsection*{change\_color}
  تابع اصلی برنامه که تغییر رنگ را انجام می‌دهد این تابع است. این تابع یک تصویر در فضایِ 
  \lr{HSV}
  و یک عدد به عنوانِ
  \lr{Target Hue}
  را ورودی می‌گیرد. بعد نمونه‌ای که داریم را به صورت سه‌تایی‌هایِ 
   $ (Hue
  ,
  Saturation
  ,
  Value) $
  \lr{reshape}
  می‌کند. بعد مینیمم و ماکزیممِ
  \lr{Hue}
  ،
  \lr{Saturation}
  و 
  \lr{Value}
  را محاسبه می‌کند تا بتواند بازه‌هایی از 
  \lr{Hue}
  ،
  \lr{Saturation}
  و 
  \lr{Value}
  که رنگِ شئ‌هایِ مدنظر ما را می‌سازند را محاسبه کند. بعد آن پیکسل‌هایی که مقادیرشون داخل این بازه‌ای که ما مشخص کرده‌ایم می‌افتد را مشخص میکنیم و در 
  \lr{frame\_threshed}
  میریزیم و بعد 
  \lr{frame\_threshed}
  را بر 255 تقسیم می‌کنیم تا مقادیر آن به صورت 0 و 1 شود.تعبیر 
  \lr{frame\_threshed}
  تا به اینجا می‌شود ماتریس که همه درایه هایِ آن
  $ (0, 0, 0) $
   است مگر جاهایی که 
  \lr{Hue}
  ،
  \lr{Saturation}
  و 
  \lr{Value}
  آن‌ها در بازه تعیین شده قرار دارد که در این نقاط مقدار
  \lr{frame\_threshed}
  سه‌تایی‌ 
  $ (1, 1, 1) $
  است. یک کپی از 
  \lr{frame\_threshed}
  می‌گیریم و 
  \lr{copy\_}
  می‌نامیم و درایه متناظر با 
  \lr{Hue}
  را در 
  \lr{copy\_} 
  صفر می‌کنیم و بعد در تصویر ورودیِ تابع ضرب می‌کنیم.تعبیر 
  \lr{copy\_}
  تا به اینجا می‌شود ماتریس که همه درایه هایِ آن
  $ (0, 0, 0) $
  است مگر جاهایی که 
  \lr{Hue}
  ،
  \lr{Saturation}
  و 
  \lr{Value}
  آن‌ها در بازه تعیین شده قرار دارد که در این نقاط مقدار
  \lr{copy\_}
  سه‌تایی‌ 
  $ (0, Saturation, Value) $
  است. و بعد با \\
  $ copy\_[:, :, 0] = target\_hue - average\_hue + input\_hsv\_image[:, :, 0] $
  مؤلفه متناظر با 
  \lr{Hue}
  در 
  \lr{copy\_}
  را با مقدار جدید جایگزین می‌کنیم. در مرحله بعد یک حلقه داریم که در آن قسمت‌هایی که با کشیدنِ مستطیل مشخص کردیم تا تغییر رنگ روی آن‌ها اعمال نشود را از نقاطی در بازه‌ی تعیین شده قرار گرفته بودند حذف می‌کنیم( با صفر کردنِ
  \lr{frame\_threshed}
  در آن نقاط
  )
  در نهایت هم با توجه به رابطه زیر، عکس خروجی در قسمت هایی که
  \lr{frame\_threshed}
  یک(1) است، می‌شود 
  \lr{copy\_}
  و در بقیه قسمت‌ها می‌شود عکس ورودی.
   \\
  $ result = np.where(frame\_threshed, copy\_, input\_hsv\_image) $
  \subsection*{$\circ$ q2.py}
  در این فایل ابتدا نام تصویرِ ورودی از کاربر گرفته ‌می‌شود، بعد تصویر لود شده و به 
  \lr{HSV}
  تبدیل می‌شود. در ادامه بعد از انجام کارهای مربط به نمایش و موس و 
  $\dots$
  در یک حلقه‌یِ بینهایت عکس و نتیجه در کنار هم نمایش داده میشود. اگر کاربر دکمه
  \lr(u)
  از کیبورد را فشار دهد 
  \lr{result}
  با فراخوانی 
  \lr{change\_color}
  به‌روزرسانی می‌شود.
  و در صورتی که کاربر 
  \lr(esc)
  را فشار دهد نتیجه ذخیره می‌شود از حلقه خارح شده و برنامه تمام می‌شود. 
  
  
  
\end{document}
