\documentclass[a4paper,12pt]{article}
\usepackage{HomeWorkTemplate}

\usepackage[utf8]{inputenc}
\usepackage[]{babel}

\setlength{\parindent}{4em}
\setlength{\parskip}{0.5em}

\renewcommand{\baselinestretch}{1.5}

\usepackage{amsmath}
\usepackage[utf8]{inputenc}
\usepackage{lmodern, textcomp}
\usepackage{circuitikz}
\usepackage[shortlabels]{enumitem}
\usepackage{hyperref}
\usepackage{tikz}
\usepackage{amsmath}
\usepackage{amssymb}
\usepackage{tcolorbox}
\usepackage{graphicx}
\usepackage{xepersian}
\settextfont{XB Niloofar}
\usetikzlibrary{arrows,automata}
\usetikzlibrary{circuits.logic.US}
\usepackage{changepage}
\newcounter{problemcounter}
\newcounter{subproblemcounter}
\setcounter{problemcounter}{1}
\setcounter{subproblemcounter}{1}
\newcommand{\problem}[1]
{
	\subsection*{
		پرسش
		\arabic{problemcounter} 
		\stepcounter{problemcounter}
		\setcounter{subproblemcounter}{1}
		#1
	}
}
\newcommand{\subproblem}{
	\textbf{\harfi{subproblemcounter})}\stepcounter{subproblemcounter}
}


\begin{document}
	\handout
	{اصول پردازش تصویر}
	{دکتر مصطفی کمالی تبریزی}
	{نیم‌سال اول 1399\lr{-}1400}
	{اطلاعیه}
	{سیدعلیرضا خادم}
	{97100398}
	{تمرین سری سوم - سوال دوم}
	زمان حدودی اجرا : 1 دقیقه و 45 ثانیه
	\section*{موارد لازم.}
	برای اجرا لازم است تا تصویر
	\lr{park.jpg}
	در مسیر
	\lr{EX3\_Q2/}
	قرار داشته باشد. همچنین در پیاده‌سازی این سوال از کتابخانه‌های 
	\lr{numpy}
	و
	\lr{cv2}
	استفاده شده است که قبل از اجرا بایستی این کتابخانه‌ها روی سیستم شما نصب باشد.
	\section*{روند کلی حل.}
	حل این سوال با استفاده از توابع موجود در کتابخانه 
	\lr{sklearn}
	صورت گرفته است. با توجه به اینکه سایز عکس برای الگوریتم 
	\lr{Mean-shift}
	بزرگ بوده و اجرای این الگوریتم روی عکسی با سایز اصلی بسیار زمان بر است، عکس را 25 برابر کوچک می‌کنیم.
	
	
	\section*{توضیح کد.}
	برنامه حاوی 1 فایل با فرمت
	\lr{.py}
	می‌باشد که توضیح آن در پایین آمده است.
	
	\subsection*{$\circ$ q2.py}
	ابتدا تصویر 
	\lr{park.jpg}
	از مسیر 
	\lr{EX3\_Q2/images}
	را در 
	\lr{src\_image}
	ذخیره می‌کنیم و بعد عکس را 25 برابر کوچک می‌کنیم. در ادامه طبق روندی که در کد با کامنت توضیح داده شده، با استفاده از توابعِ 
	\lr{sklearn.cluster}
	الگوریتم را روی عکس 25 برابر کوچک شده اجرا می‌کنیم و در نهایت نتیجه را با نام 
	\lr{res04.jpg}
	در مسیر 
	\lr{EX3\_Q2/results/}
	ذخیره می‌کنیم.
	
	
	
\end{document}
