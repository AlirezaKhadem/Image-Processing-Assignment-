\documentclass[a4paper,12pt]{article}
\usepackage{HomeWorkTemplate}

\usepackage[utf8]{inputenc}
\usepackage[]{babel}

\setlength{\parindent}{4em}
\setlength{\parskip}{0.5em}

\renewcommand{\baselinestretch}{1.5}

\usepackage{amsmath}
\usepackage[utf8]{inputenc}
\usepackage{lmodern, textcomp}
\usepackage{circuitikz}
\usepackage[shortlabels]{enumitem}
\usepackage{hyperref}
\usepackage{tikz}
\usepackage{amsmath}
\usepackage{amssymb}
\usepackage{tcolorbox}
\usepackage{graphicx}
\usepackage{xepersian}
\settextfont{XB Niloofar}
\usetikzlibrary{arrows,automata}
\usetikzlibrary{circuits.logic.US}
\usepackage{changepage}
\newcounter{problemcounter}
\newcounter{subproblemcounter}
\setcounter{problemcounter}{1}
\setcounter{subproblemcounter}{1}
\newcommand{\problem}[1]
{
	\subsection*{
		پرسش
		\arabic{problemcounter} 
		\stepcounter{problemcounter}
		\setcounter{subproblemcounter}{1}
		#1
	}
}
\newcommand{\subproblem}{
	\textbf{\harfi{subproblemcounter})}\stepcounter{subproblemcounter}
}


\begin{document}
	\handout
	{اصول پردازش تصویر}
	{دکتر مصطفی کمالی تبریزی}
	{نیم‌سال اول 1399\lr{-}1400}
	{اطلاعیه}
	{سیدعلیرضا خادم}
	{97100398}
	{تمرین سری دوم - سوال اول}
	\section*{موارد لازم.}
	برای اجرا لازم است تا تصویر
	\lr{flowers\_blur.jpg}
	در مسیر
	\lr{EX2\_Q1/images/}
	قرار داشته باشد.
	\section*{روند کلی حل.}
	ایده اصلی و انتخاب ماتریس برای 
	\lr{sharpening}
	برگرفته از جلسه ویدئو جلسه 7ام درس و صفحلات 13، 14 و 15 اسلاید‌های جلسه 7ام درس می‌باشد. روندی که برای محاسبه بهترین 
	\lr{alpha}
	به عنوان ضریب
	\lr{unsharp\_image}
	در تابع 
	\lr{unsharp}
	طی شده است به صورت تجربی و بر پایه الگوریتم باینری سرچ بوده. به این شکل که در ابتدا به ازای
	\lr{alpha = 1}
	،
	\lr{alpha = 0.5}
	و
	\lr{alpha = 0}
	همانطور که تصاویر در مسیر 
	\lr{EX2\_Q1/images/}
	مشاهده است مقایسه انجام شده و نتیجه گرفته شد که
	\lr{alpha}
	مناسب مقداری بین
	\lr{1}
	و 
	\lr{0.5}
	دارد و .... . تا در نهایت به 
	\lr{ahpha = .625}
	رسیدیم. 
	
	\section*{توضیح کد.}
	برنامه در مجموع حاوی 2 فایل با فرمت
	\lr{.py}
	می‌باشد که توضیحات هر فایل در پایین آمده است.
	\subsection*{$\circ$ utilities.py}
	\subsubsection*{apply\_unsharp\_filter(input\_channel)}
	این  تابع 
	\lr{input\_channel}
	را به عنوان ورودی می‌گیرد و ماتریسی که در پایین آمده را با 
	\lr{input\_channel}
	کانوالو می‌کند و به عنوان خروجی بر می‌گرداند.\\
	\begin{align*}
		\begin{bmatrix}
			-1 & -1 & -1\\
			-1 & 8  & -1\\
			-1 & -1 & -1
		\end{bmatrix}
	\end{align*}
	
	\subsubsection*{unsharp(src\_image)}
	تابع 
	\lr{unsharp}
	یک تصویر به عنوان ورودی می‌گیرد  سپس با استفاده از تابع 
	\lr{cv.split}
	آن را به کانال‌های 
	\lr{B}
	،
	\lr{G}
	و
	\lr{R}
	تجزیه می‌کند. در نهایت بعد از اینکه فیلتر 
	\lr{unsharp}
	با استفاده از تابع 
	\lr{apply\_unsharp\_filter}
	روی کانال‌های مختلف اعمال شد، با استفاده از تابع 
	\lr{cv.merge}
	، 3 کانال 
	\lr{unsharp}
	شده را با هم مرج می‌کند تا تصویرِ
	\lr{unsharp}
	شده متناظر با 
	\lr{src\_image}
	حاصل شود و به عنوان خروجی باز گرداند.
	\subsubsection*{\lr{sharp(src\_image, alpha)}}
	این تابع تصویر 
	\lr{src\_image}
	که به عنوان ورودی گرفته است را به تابع 
	\lr{unsharp}
	میده تا 
	\lr{unsharp}
	شده آن را به دست آورد. سپس تصویر 
	\lr{src\_image}
	شده را با ضریب 
	\lr{alpha}
	از تصویر 
	\lr{unsharp\_image}
	جمع می‌کند و به عنوان خروجی برمی‌گرداند.
	
	\subsection*{$\circ$ q1.py}
	در این فایل ابتدا تصویر 
	\lr{flowers\_blur.jpg}
	از مسیر 
	\lr{EX2\_Q1/images/}
	لود می‌شود سپس با استفاده از توابع 
	\lr{unshar(src\_image)}
	و
	\lr{sharp(src\_image, .625)}
	تصاویر 
	sharp
	و 
	\lr{unsharp}
	محاسبه می‌شوند و در نهایت به ترتیب با نام‌های 
	\lr{res02.jpg}
	و
	\lr{res01.jpg}
	در مسیر
	\lr{EX2\_Q1/results/}
	ذخیره می‌شوند.
	
	
\end{document}
